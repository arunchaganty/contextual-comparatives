\section{Introduction}

% motivation
When posed with a numeric mention, such as ``Gasoline inventories [\dots] expected a \textit{1 million barrel} increase'' (\reffig{overview}), it is often difficult to comprehend the scale of large (and small) absolute values like \textit{1 million barrels} \citep{paulos1988innumeracy,seife2010proofiness}. 
Studies have repeatedly shown\needcite{} that providing relative comparisons, such as ``about 1/5th the oil produced in the world in the time it takes for a basketball game'' significantly improves comprehension, e.g.\ when measured in terms of memory retention or outlier detection\cite{barrio2016comprehension}.
% HCI work
Indeed, previous work in the HCI community has 
However, this is limited in the range of facts it can express while still using a small database of ground truths.

In this work, we leverage compositionality to achieve broad coverage from a relatively small collection of relevant facts.
In doing so, we must address two key challenges: selecting familiar, relevant and easy-to-understand expressions and generating easy-to-digest perspectives.
We tackle the first challenge using a search guided by supervised data --- signals from proximity, similairty, compositionality.
We treat the problem of generation as a translation problem and use a neural sequence-to-sequence model that supports copying to generate perspectives from numeric expressions.

We evaluate our work on 30,000 numeric mentions collected from different newswire corpora and generate perspectives from a manually curated database of 180 facts.
The turkers love us. \citep{manning99nlp}

\Fig{figures/overview}{1}{overview}{Overview diagram}
